% LaTeX Template for short student reports.
% Citations should be in bibtex format and go in references.bib
\documentclass[a4paper, 11pt]{article}
\usepackage[top=3cm, bottom=3cm, left = 2cm, right = 2cm]{geometry} 
\geometry{a4paper} 
\usepackage[utf8]{inputenc}
\usepackage{textcomp}
\usepackage{graphicx} 
\usepackage{amsmath,amssymb}  
\usepackage{bm}  
\usepackage[pdftex,bookmarks,colorlinks,breaklinks]{hyperref}  
%\hypersetup{linkcolor=black,citecolor=black,filecolor=black,urlcolor=black} % black links, for printed output
\usepackage{memhfixc} 
\usepackage{pdfsync}  
\usepackage{fancyhdr}
\usepackage{enumitem}
\usepackage{indentfirst}
\usepackage{dsfont}
\pagestyle{fancy}

\title{Ordinal Regression}
\author{
  Rahul Ramachandran\\
  \texttt{cs21btech11049}
  \and
  Rishit D\\
  \texttt{cs21btech11053}
}
%\date{}

\begin{document}
\maketitle
\tableofcontents

\section{Ordinal Regression: Summary}
% Ordinal random variables are based on classfifcation random variables but with a stochastic ordering. For instance, support for various political parties is considered a classfication problem whereas support for various parts of the political spectrum is an ordinal regression problem.
% Consider $k$ categories which are the qualitative representation of the ordering. We would want to determine the probabilities of landing in various categories based on a set of independent features.
% \newline
% Assume the independent variables to be represented through $\textbf{x}$ with the probability of being in category $i$ as $\pi_i(\textbf{x})$. Now define cumulative probabilities $\gamma_i(\textbf{x})$ as follows: \newline
% \begin{align}
%     \gamma_i(\textbf{x}) = \sum_{a=1}^{i} \pi_a(\textbf{x})
% \end{align}

% Our ordinal regression model is based on the General Linearized Model, represented as follows:
% \begin{align}
%     g(\gamma_i(\textbf{x})) = \theta_i - \beta^T\textbf{x} 
% \end{align}

% where $\theta_i$ represents the cut-point for category $i$, $\beta$ represents a vector of unknown parameters, $\textbf{x}$ represents the vector of independent features and $g(x)$ is the link function for the GLM.
% Note that link functions are monotone and map $(0, 1)$ to $(-\infty, \infty)$. \newline

% Intuitively, we note that our set of cut-point $\{\theta_i\}$ must be monotone. One can note that we obtain a set of parallel hyperplanes that separate out different categories when we consider our link function as the identity function.
% But based on the nature of the dataset given to us (in specific, the possible distribution from which the outcome $Y$ may be generated), we choose different link functions and obtain parameters. One may also note that we only have a single set of unknown paramaters $\beta$ common to all categories, thus geneereating the set of parallel hyperplanes. \newline

% Now consider two input vectors \textbf{$x_1$} and \textbf{$x_2$}. Then we have,
% \begin{align}
%     g(\gamma_i(\textbf{$x_1$})) - g(\gamma_i(\textbf{$x_2$})) = \beta^T(\textbf{$x_2$} - \textbf{$x_1$})
% \end{align}
% We observe that the right hand side is completely independent of the category. Hence, by monotonicity of the link function $g$, we have that either of the following two equations to hold.
% \begin{align}
%     \gamma_i(\textbf{$x_1$}) > \gamma_i(\textbf{$x_2$}) \\
%     \gamma_i(\textbf{$x_1$}) < \gamma_i(\textbf{$x_2$})
% \end{align}
% for all categories $i$.

% \subsection{Proportional Odds Model}
% Assume that odds of $\textbf{x}$ belonging to a category $i$ such that $i < j$ is modelled as:
% \begin{align} 
%     T_j(\textbf{x}) = C_j e^{-\beta^T\textbf{x}}
% \end{align}

% This model is termed the proportional odds model as the ratio of odds of $i < j$ when input vectors are \textbf{$x_1$} and \textbf{$x_2$} is independent of the category.
% \begin{align}
%     \frac{T_j(\textbf{$x_1$})}{T_j(\textbf{$x_2$})} = e^{\beta^T(\textbf{$x_2$} - \textbf{$x_1$})}
% \end{align}

% We observe that applying $log$ on either side of the proportional odds model equation yields our linear
Ordinal random variables are based on classfifcation random variables but with a stochastic ordering. For instance, support for various political parties is considered a classfication problem whereas support for various parts of the political spectrum is an ordinal regression problem.
Consider $k$ categories which are the qualitative representation of the ordering. We would want to determine the probabilities of landing in various categories based on a set of independent features.
\newline
Assume the independent variables to be represented through $\textbf{x}$ with the probability of being in category $i$ as $\pi_i(\textbf{x})$. Now define cumulative probabilities $\gamma_i(\textbf{x})$ as follows: \newline
\begin{align}
    \gamma_i(\textbf{x}) = \sum_{a=1}^{i} \pi_a(\textbf{x})
\end{align}

Our ordinal regression model is based on the General Linearized Model, represented as follows:
\begin{align}
    g(\gamma_i(\textbf{x})) = \theta_i - \beta^T\textbf{x} 
\end{align}

where $\theta_i$ represents the cut-point for category $i$, $\beta$ represents a vector of unknown parameters, $\textbf{x}$ represents the vector of independent features and $g(x)$ is the link function for the GLM.
Note that link functions are monotone and map $(0, 1)$ to $(-\infty, \infty)$. \newline

\subsection{Examples}
For instance, defining the link function $g(x)$ as the logit function would give the proportional odds model as follows:
\begin{align}
    ln\left(\frac{\gamma_i(\textbf{x})}{1 - \gamma_i(\textbf{x})}\right) = \theta_i - \beta^T\textbf{x}
\end{align}
Note that the ratio of two odds for two input vectors \textbf{$x_1$} and \textbf{$x_2$} is independent of the category $i$, hence the name.

Similarly, defining the link function $g(x)$ as the complementary log-log function would generate the proportional hazards model, described as follows:
\begin{align}
    ln(-ln(1 - \gamma_i(\textbf{x}))) = \theta_i - \beta^T\textbf{x}
\end{align}
Here, defining the survivor function $S(t;x)$ as $1 - \gamma_i(\textbf{x})$ would ensure that the ratio of the survivor functions remains constant and is independent of the category, giving us the proportional hazard model.

\subsection{Differences with Multiclass Classification} 
Consider a multinomial logistic regression problem where the probability of sample \textbf{$x$} residing in category $i$, captured by the output variable $Y$.
\begin{align}
    \pi_i(\textbf{x}) = p(Y = i|\textbf{x}) = \frac{e^{\beta_i^T\textbf{x}}}{\sum_{a=1}^{k} e^{\beta_a^T\textbf{x}}}
\end{align}
One may note that the $\textbf{x}= (1, x_1, ..., x_D)^T$ here also includes $1$ as its first entry and correspondingly the first term of the parameter vector $\beta = (\beta_0, \beta_1, ..., \beta_D)^T$ is the intercept term, $\beta_0$. \newline

Firstly, the ordinal regression model uses cumulative probabilities $\gamma_i(\textbf{x})$ rather than the probability masses $\pi_i(x)$ used in the softmax model. \newline

Secondly, we observe the linear estimators 
\begin{align}
    \eta_i = \beta_i^T\textbf{x}
\end{align}

for each case. In the case of the softmax model, every set of parameters $\beta_i$ for category $i$ is not necessarily the same.
But in case of the ordinal regression model, we have
\begin{align}
    \eta_i = \beta_{i0} + \beta_{i1}\textbf{$x_{1}$} + ... + \beta_{iD}\textbf{$x_{D}$} = \theta_i - \beta_*^T\textbf{x}
\end{align}
where we notice that $\beta_{i0}$ serves the role of the intercept $\theta_i$ and that $\beta_*$ remains constant over all categories $i$. Hence, we only have $D + k$ parameters as compared to $(D+1)k$ parameters in a multinomial logistic regression.
Since our $\beta_*$ remains constant over all categories, the linear estimators $\eta_i$ form a set of parallel hyperplanes as opposed to a multinomial logistic regression where there are no such constraints placed on hyperplanes to distinguish samples. 
Intuitively, this is logical as ordinal regression must maintain stochastic ordering. Consider two hyperplanes intersecting in an ordinal regression setting; this would imply a switch in ordering at some point \textbf{$x_0$} in the input space, violating the ordinality of categories.
\newline

\subsection{Log Likelihood and Log Odd Differences}
In both cases, given $n$ input points $\{{$\textbf{x_i}$}\}_{1..n}$, the likelihood function would be given as:
\begin{align}
    \mathcal{L} &= p(Y|\beta_1, ..., \beta_k) \\
    &= \prod_{i=1}^n \prod_{j=1}^k \pi_j({x_n})^{\mathds{1}\{j=Y_i\}}
\end{align}

Similarly, the log likelihood $l$ would be defined as:
\begin{align}
    l &= ln\left(\mathcal{L}\right) \\
    &= \sum_{i=1}^n \sum_{j=1}^k \delta_{j,Y_i} ln\left(\pi_j({x_n})\right)
\end{align}

For multiclass logistic regression, $l$ takes the following form:
\begin{align}
    l = \sum_{i=1}^n \sum_{j=1}^k \delta_{j,Y_i} \left(\beta_jx_i - ln\left(\sum_{j=1}^k e^{\beta_j^T{x_i}}\right)\right)
\end{align}

Assuming $g(x)$ is the link function for ordinal regression models and using the fact that $\pi_j(\textbf{x}) = \gamma_j(\textbf{x}) - \gamma_{j-1}(\textbf{x})$, we get the following expression for $l$:
\begin{align}
    l = \sum_{i=1}^n \sum_{j=1}^k \delta_{j,Y_i} \left(ln(g^{-1}(\theta_j - \beta_*^T(x_i)) - g^{-1}(\theta_{j-1} - \beta_*^T(x_i)))\right)
\end{align}

When using proportional odds model, the above expression reduces to:
\begin{align}
    l = \sum_{i=1}^n \sum_{j=1}^k \delta_{j,Y_i} \left(ln\left(\frac{e^{\theta_j}}{e^{\theta_j} + e^{\beta_*^Tx_i}} - \frac{e^{\theta_{j-1}}}{e^{\theta_{j-1}} + e^{\beta_*^Tx_i}}\right)\right)
\end{align}

In case of multiclass classfication, our log-odds are defined as follows:
\begin{align}
    ln\left( \frac{\pi_i(\textbf{x})}{1 - \pi_i(\textbf{x})} \right) = \beta_i^T\textbf{x} - \ln\left( \sum_{j\neqi}e^{\beta_j^T\textbf{x}} \right)
\end{align}

But in the case of ordinal regression (in this case, specifically proportional odds), we define log-odds in terms of cumulative probabilities $\gamma_i(\textbf{x})$ as:
\begin{align}
    ln\left( \frac{\gamma_i(\textbf{x})}{1 - \gamma_i(\textbf{x})} \right) &= \ln\left( \frac{g^{-1}(\theta_j - \beta_*^Tx_i)}{1 - g^{-1}(\theta_j - \beta_*^Tx_i)}\right) \\
    &= \theta_j - \beta_*^Tx_i
\end{align}


\section{Likelihood \& Parameter Estimation}
In all of our following derivations, we consider the proportional-odds model. Consider a single input vector \textbf{x} and its associated output $Y = (n_1, n_2, ..., n_k)^T$. Define $R_i$ and $Z_i$ as following:
\begin{align}
    R_i &= \sum_{j=1}^i n_i \\
    Z_i &= \frac{R_i}{N} 
\end{align}
where $N = \sum_{j=1}^k n_i$. \newline

We interpret our partial likelihood for ordinal regression as follows (we drop the \textbf{x} for convenience):
\begin{align}
    \mathcal{L(\textbf{x})} &= \prod_{i=1}^k \pi_i(\textbf{x})^n_i \\
    &= \gamma_1(\textbf{x})^{n_1} \prod_{i=2}^k (\gamma_i(\textbf{x}) - \gamma_{i-1}(\textbf{x}))^{n_i} \\
    &= \gamma_1(\textbf{x})^{R_1} \prod_{i=2}^k (\gamma_i(\textbf{x}) - \gamma_{i-1}(\textbf{x}))^{R_i - R_{i-1}} \\
    &= \left( \prod_{i=2}^k \frac{\gamma_{i-1}^{R_{i-1}(\textbf{x})} \times (\gamma_i(\textbf{x}) - \gamma_{i-1}^{R_i - R_{i-1}}(\textbf{x}))}{\gamma_{i}^{R_{i}}(\textbf{x})} \right) \gamma_k^{R_k}(\textbf{x})\\
    &= \prod_{i=2}^k \frac{\gamma_{i-1}^{R_{i-1}(\textbf{x})} \times (\gamma_i(\textbf{x}) - \gamma_{i-1}^{R_i - R_{i-1}}(\textbf{x}))}{\gamma_{i}^{R_{i}}(\textbf{x})} \\
    &= \prod_{i=2}^k \left(\frac{\gamma_{i-1}}{\gamma_i}\right)^{R_{i-1}} \left(\frac{\gamma_i - \gamma_{i-1}}{\gamma_i}\right)^{R_i - R_{i-1}} \\
    &= \prod_{i=2}^k \left(\frac{\gamma_{i-1}}{\gamma_i - \gamma_{i-1}}\right)^{R_{i-1}} \left(\frac{\gamma_i - \gamma_{i-1}}{\gamma_i}\right)^{R_i}
\end{align}

Our partial log-likelihood would be defined as:
\begin{align}
    l(\textbf{x}) &= \sum_{i=2}^k R_{i-1}ln\left(\frac{\gamma_{i-1}}{\gamma_i - \gamma_{i-1}}\right) + R_iln\left(\frac{\gamma_i - \gamma_{i-1}}{\gamma_i}\right) \\
    &= \sum_{i=2}^k R_{i-1}\phi_{i-1} - R_i f(\phi_{i-1}) \\
    &= N \left( \sum_{i=2}^k Z_{i-1}\phi_{i-1} - Z_i f(\phi_{i-1}) \right) \\
\end{align}
where we define:
\begin{align}
    \phi_{i} &= ln\left(\frac{\gamma_{i}}{\gamma_{i+1} - \gamma_{i}}\right) \\
    f(t) &= ln(1 + e^t)
\end{align}

Consider the gradient of $l(\textbf{x})$ with respect to $\beta_*$ which can be computed as follows:
\begin{align}
    \nabla_{\beta_*}l(\textbf{x}) &= N \left( \sum_{i=2}^k Z_{i-1} \nabla_{\beta_*}\phi_{i-1} - Z_i \nabla_{\beta_*}f(\phi_{i-1}) \right) \\
    &= N \left( \sum_{i=2}^k \left( Z_{i-1} - Z_i f'(\phi_{i-1}) \right) \nabla_{\beta_*}\phi_{i-1} \right) \\
    &= N \left( \sum_{i=2}^k \left( Z_{i-1} - Z_i \frac{\gamma_{i-1}}{\gamma_i} \right) \gamma_{i}\textbf{x} \right) \\
    &= N \left( \sum_{i=2}^k \left( \gamma_i Z_{i-1} - \gamma_{i-1} Z_i \right) \textbf{x} \right)
\end{align}

Assume we have $P$ input points \textbf{$x_p$.} Now, our gradient of our final log-likelihood function would be:
\begin{align}
    \nabla_{\beta_*}l &= \sum_{p=1}^P\nabla_{\beta_*}l(\textbf{$x_p$}) \\
    &= \sum_{p=1}^P N_p \left( \sum_{i=2}^k \left( \gamma_{p,i} Z_{p,i-1} - \gamma_{p,i-1} Z_{p,i} \right) \textbf{$x_p$} \right) \\
    &= \sum_{i=2}^k \left( \gamma_{i} \mathds{E}(Z_{i-1}) - \gamma_{i-1} \mathds{E}(Z_i) \right) (????????)
\end{align}

\subsection{Other Derivations}
We derive $\nabla_{\beta_*} \gamma_j$ as follows:
\begin{align}
    ln\left(\frac{\gamma_j}{1 - \gamma_j}\right) &= \theta_j - \beta_*^T\textbf{x} \\
    \nabla_{\beta_*} \left( ln\left(\frac{\gamma_j}{1 - \gamma_j}\right) \right) &= \nabla_{\beta_*} \left( \theta_j - \beta_*^T\textbf{x} \right) \\
    \left( \frac{1 - \gamma_j}{\gamma_j} \right) \left( \frac{(1 - \gamma_j) - (-\gamma_j)}{(1 - \gamma_j)^2} \right) \nabla_{\beta_*}\gamma_j &= \textbf{x} \\
    \nabla_{\beta_*}\gamma_j &= \gamma_j(1 - \gamma_j)\textbf{x}
\end{align}

We derive $f'(t)$ as:
\begin{align}
    f(t) &= ln(1+e^t) \\
    f'(t) &= \frac{e^t}{1+e^t}
\end{align}

Note we can derive the following partial derivatives:
\begin{align}
    \frac{\partial \phi_j}{\partial \gamma_j} &= \frac{\gamma_{j+1}}{\gamma_j(\gamma_{j+1} - \gamma_j)} \\
    \frac{\partial \phi_j}{\partial \gamma_{j+1}} &= \frac{-1}{\gamma_{j+1} - \gamma_j}
\end{align}

Using the above we can calculate $\nabla_{\beta_*} \phi_j$ through exact differentials as follows:
\begin{align}
    \nabla_{\beta_*}(\phi_j) &= \frac{\partial \phi_j}{\partial \gamma_j}\nabla_{\beta_*}(\gamma_j) + \frac{\partial \phi_j}{\partial \gamma_{j+1}}\nabla_{\beta_*}(\gamma_{j+1}) \\
    &= \gamma_{j+1}\textbf{x}
\end{align}

\bibliographystyle{abbrv}
% \bibliography{references}  % need to put bibtex references in references.bib 
\end{document}